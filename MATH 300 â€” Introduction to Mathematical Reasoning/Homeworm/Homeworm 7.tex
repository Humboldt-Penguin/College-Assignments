%%%%%%%%%%%%%%%%%%%%%%%%%%%%%%%%%%%%%%%%%%%%%%%%%%%%%%%%%%%%%%%%%%%%%%%%%%%%%%%%%
\documentclass{article}
% Feel free to contact me for any reason:
%% Email 1: zain.kamal@rutgers.edu
%% Email 2: z.kamal2021@gmail.com
%% Discord: alci#6038

% Last updated 1/28/22

% Template based off of Justin Kim's (don't know where he got it from originally), but I've made a ton of edits. His soul lives on in random packages and commands I'm too lazy to comment out.


%%%%%%%%%%%%%%%%%%%%%%%%%%%%%%%%%%%%%%%%%%%%%%%%%%%%%%%%%%%%%%%%%%%%%%%%%%%%%%%%%%%%
%%% Default packages:

\usepackage[margin=1in]{geometry} 
\usepackage{amsmath,amsthm,amssymb,amsfonts, fancyhdr, color, comment, graphicx, environ}
\usepackage{xcolor}
\usepackage{mdframed}
\usepackage{bm}
\usepackage[shortlabels]{enumitem}
\usepackage{mathtools}
\usepackage{listings}
\usepackage{stmaryrd}
\usepackage{indentfirst}
\usepackage{hyperref}


%%%%%%%%%%%%%%%%%%%%%%%%%%%%%%%%%%%%%%%%%
%%% Pacakages I've added:

%% For Math 300 (Winter 2021):

\usepackage{mathdots} % for \iddots

\usepackage[ruled]{algorithm2e} % Algorithms
% NOTE: FIND A BETTER ALGORITHM PACKAGE, OR ATLEAST LEARN HOW TO USE THIS ONE BECAUSE DOUBLE INDENTING IS A FUCKING NIGHTMARE (or just import from mathcha?)
%% Example algorithm:
% \begin{center}
% 	\begin{minipage}{0.5\linewidth} % Adjust the minipage width to accomodate for the length of algorithm lines
% 		\begin{algorithm}[H]
% 			\KwIn{$(a, b)$, two floating-point numbers}  % Algorithm inputs
% 			\KwResult{$(c, d)$, such that $a+b = c + d$} % Algorithm outputs/results
% 			\medskip
% 			\If{$\vert b\vert > \vert a\vert$}{
% 				exchange $a$ and $b$ \;
% 			}
% 			$c \leftarrow a + b$ \;
% 			$z \leftarrow c - a$ \;
% 			$d \leftarrow b - z$ \;
% 			{\bf return} $(c,d)$ \;
% 			\caption{\texttt{FastTwoSum}} % Algorithm name
% 			\label{alg:fastTwoSum}   % optional label to refer to
% 		\end{algorithm}
% 	\end{minipage}
% \end{center}



%%%%%%%%%%%%%%%%%%%%%%%%%%%%%%%%%%%%%%%%%%%%%%%%%%%%%%%%%%%%%%%%%%%%%%%%%%%%%%%%%%%%
%%% Default commands:

\renewcommand{\vec}[1]{\mathbf{#1}}
	
\newcommand{\WidestEntry}{$lon_1$}%
\newcommand{\SetToWidest}[1]{\makebox[\widthof{\WidestEntry}]{#1}}%
\newcommand\tab[1][0.61cm]{\hspace*{#1}}
\newcommand{\nats}{\mathbb{N}}
\newcommand{\rats}{\mathbb{Q}}
\newcommand{\reals}{\mathbb{R}}
\newcommand{\Z}[1]{\mathbb{Z}_{#1}}
\newcommand{\BigO}[1]{\mathcal{O}(#1)}
\newcommand{\seq[1]}{(#1_n)}
\newcommand{\subseq[1]}{(#1_{n_k})}
\newcommand{\Lim}[2]{\lim \limits _{#1 \to #2}}
\newcommand{\Min}[2]{\min \{#1, #2\}}
\newcommand{\inv}{^{-1}}
\newcommand{\h}{^\text{th}}
\newcommand{\lrangle}[1]{\langle #1 \rangle}
\newcommand{\abs}[1]{\left\lvert #1 \right\rvert}

\DeclarePairedDelimiter{\ceil}{\lceil}{\rceil}
\DeclarePairedDelimiter{\floor}{\lfloor}{\rfloor}
\DeclareMathOperator{\supp}{supp}
\DeclareMathOperator{\rad}{rad}
\DeclareMathOperator*{\argmin}{arg\,min}
\DeclareMathOperator*{\argmax}{arg\,max}
\DeclareMathOperator*{\Var}{Var}
\DeclareMathOperator*{\Cov}{Cov}
\DeclareMathOperator*{\Corr}{Corr}
\DeclareMathOperator*{\Aut}{Aut}
\newcommand{\prob}[1]{\section*{Problem #1}}


%%%%%%%%%%%%%%%%%%%%%%%%%%%%%%%%%%%%%%%%%
%%% Commands I've added:

%% For Math 300 (Winter 2021):

\newcommand{\lrbrace}[1]{\{ #1 \}}
\newcommand{\powerset}{\mathcal{P}}
\newcommand{\ints}{\mathbb{Z}}

% Source/inspiration: https://tex.stackexchange.com/a/42728:
\newcommand{\numberthis}{\addtocounter{equation}{1}\tag{\theequation}\label{\theequation}}
    % Within an `align*` environment, put `\numberthis` after a line to number it. 
    % Access it with `\eqref{ [number of equation] }`
\newcommand{\numberthiswith}[1]{\addtocounter{equation}{1}\tag{\theequation}\label{#1}}
    % Within an `align*` environment, put `\numberthiswith{ [your_label] }` after a line to number it. 
    % Access it with `\eqref{ [your_label] }`



%%%%%%%%%%%%%%%%%%%%%%%%%%%%%%%%%%%%%%%%%%%%%%%%%%%%%%%%%%%%%%%%%%%%%%%%%%%%%%%%%%%%
%%% Default formatting:

\hypersetup{
    colorlinks=true,
    linkcolor=blue,
    filecolor=magenta,      
    urlcolor=blue,
}

\setlength{\parindent}{0cm}
\setlength{\parskip}{6pt}

\pagestyle{fancy}


%% Misc formatting additions

% make bullets with itemize much smaller
\newlength{\mylen}
\setbox1=\hbox{$\bullet$}\setbox2=\hbox{\tiny$\bullet$}
\setlength{\mylen}{\dimexpr0.5\ht1-0.5\ht2}
\renewcommand\labelitemi{\raisebox{\mylen}{\tiny$\bullet$}}


% Modified version of problem environment below
% \newenvironment{problem}[2][Problem]
%     { \begin{mdframed}[backgroundcolor=gray!5] \textbf{#1 #2} \\}
%     {  \end{mdframed}}
% \newenvironment{solution}{\textbf{Solution}\\}


%%%%%%%%%%%%%%%%%%%%%%%%%%%%%%%%%%%%%%%%%
%%% Formatting I've added:

%% Grey boxes for problem statements (note that I don't have a "solution" section):

% Problem environment, but shows "(a)" instead of "Problem a"
\newenvironment{problem}[2][]
    { \begin{mdframed}[backgroundcolor=gray!5] \textbf{#1 (#2)}}
    {  \end{mdframed}}
% Problem environment, but no "([input_char])" at all
\newenvironment{problem*}
    { \begin{mdframed}[backgroundcolor=gray!5] \\}
    {  \end{mdframed}}


% Example environment, currently identical to "problem" (Note: this is better written than the problem environment because I wrote it myself from scratch. Use this as an example for future new environments.)
\newcounter{example}[section]
\newenvironment{example}
    { 
        \refstepcounter{example}
        \begin{mdframed}[backgroundcolor=gray!5]
        \textbf{\\Example \thesection.\theexample:}
    }
    {\\ \end{mdframed}}


%%%%%%%%%%%%%%%%%%%%%%%%%%%%%%%%%%%%%%%%%%%%%%%%%%%%%%%%%%%%%%%%%%%%%%%%%%%%%%%%%%%%
% Misc things I've added




%%%%%%%%%%%%%%%%%%%%%%%%%%%%%%%%%%%%%%%%%%%%%
% Fill in the appropriate information below
\lhead{Zain Kamal}
% \rhead{Math 244 Spring 2022} % Moved to document
% \chead{\textbf{Homework 2}} % Moved to document
\begin{document}
\chead{\textbf{Homeworm 7}}
%%%%%%%%%%%%%%%%%%%%%%%%%%%%%%%%%%%%%%%%%%%%%%%%%%%%%%%%%%%%%%%%%%%%%%%%%%%%%%%%%
\section{}

\begin{problem*}
Let $A = \{1, 2\}, B = \{x, y\}$. List all functions from $A \to B$ that are:
\end{problem*}

\begin{problem} a
injections;
\end{problem}


\begin{itemize}
\item $\{( 1,x) ,( 2,y)\}$
\item $\{( 1,y) ,( 2,x)\}$
\end{itemize}


\begin{problem} b
surjections; 
\end{problem}

same as (a)

\begin{problem} c
bijections; 
\end{problem}

same as (a)

\begin{problem} d
none of the above.
\end{problem}


\begin{itemize}
\item $\{( 1,x) ,( 2,x)\}$
\item $\{( 1,y) ,( 2,y)\}$
\end{itemize}

\
\hline
%%%%%%%%%%%%%%%%%%%%%%%%%%%%%%%%%%%%%%%%%%%%%%%%%%%%%%%%%%%%%%%%%%%%%%%%%%%%%%%%%
\section{}

\begin{problem} a
Supply a proof for Proposition 7.8.
\end{problem}

Choose some $x,y\in A$ such that, for some $z\in C$, $g( f( x)) =g( f( y)) =z$. 

Let $a=f( x)$ and $b=f( y)$. Because $g$ is injective, $g( a) =g( b)$ implies $a=b$. Therefore $f( x) =f( y)$. Because $f$ is injective, $f( x) =f( y)$ implies $x=y$. 

Thus $g\circ f$ is injective, because for all $x,y\in A$, if $g( f( x)) =g( f( y))$ then $x=y$.\qed 

\

\begin{problem} b
Suppose $f:A\to B$ and $g:B\to C$. If $g\circ f$ is injective, is it necessarily the case that $f$ is injective? Is it necessarily the case that $g$ is injective? Prove or disprove your claims.
\end{problem}

Choose some $x,y\in A$ such that $f( x) =f( y)$. We know $f( x) ,f( y) \in B$, so we can take $g$ of both sides to find $g( f( x)) =g( f( y))$. As proven in part (a), since $g\circ f$ is injective, this implies $x=y$. Thus $f$ is injective. 

If we let $a=f( x)$ and $b=f( y)$, then the previous equation ($g( f( x)) =g( f( y))$) becomes $g( a) =g( b)$. Since $x=y$, we also know $f( x) =f( y)$, or $a=b$. Note that $a,b\in f( A)$, thus $g$ is injective over the image of $A$ under $f$.\qed

\

\begin{problem} c
Repeat part (b) but replace injective with surjective. 
\end{problem}

Choose some $z\in C$. Since $g\circ f$ is surjective, there exists an $x\in A$ such that $( g\circ f)( x) =g( f( x)) =z$. If we let $b=f( x)$, then we get $g( y) =z$. Thus $g$ is surjective.

However $f$ is not necessarily surjective as the image of $f$ does not have to be all of $B$. For example, take $A=C=\{1\}$, $B=[ 2]$, and $f( x) =g( x) =1$.

\
\hline
%%%%%%%%%%%%%%%%%%%%%%%%%%%%%%%%%%%%%%%%%%%%%%%%%%%%%%%%%%%%%%%%%%%%%%%%%%%%%%%%%
\section{}

\begin{problem*}
Find a function from $\mathbb{R}$ to $\mathbb{R}$, and supply a proof of your claim, that is:
\end{problem*}

\begin{problem} a
an injection but not a surjection;
\end{problem}

$f( x) =e^{x}$ is an injection but not a surjection. 

Injective: Choose some $a,b\in \mathbb{R}$ such that $f( a) =f( b)$. Some algebraic manipulation yields:
\begin{align*}
f( a) & =f( b)\\
e^{a} & =e^{b}\\
\ln e^{a} & =\ln e^{b}\\
a & =b.
\end{align*}
Not surjective: As an example, $-1\in \mathbb{R}$, but there does not exist some $x\in \mathbb{R}$ such that $e^{x} =-1$. The range of $f( x)$ is $( 0,\infty )$.



\

\begin{problem} b
a surjection but not an injection;
\end{problem}

$f( x) =x^{3} +x^{2}$ is a surjection but not an injection.

Surjective: We know $f$ is continuous for $\mathbb{R}$ and $\lim _{x\rightarrow -\infty } f( x) =-\infty $ and $\lim _{x\rightarrow \infty } f( x) =\infty $. Therefore, by the intermediate value theorem, for any $y\in ( -\infty ,\infty ) =\mathbb{R}$, there exists an $x\in \mathbb{R}$ such that $f( x) =y$.

Not Injective: As an example, if $a=0$ and $b=-1$, we can have $f( a) =f( b) =0$ while $a\neq b$.

\

\begin{problem} c
a bijection; 
\end{problem}

$f( x) =x$ is a bijection.

Injective: If we choose some $a,b\in \mathbb{R}$ such that $f( a) =f( b)$, it follows that $a=b$.

Surjective: Choose some $y\in \mathbb{R}$. There exists some $x=y$ such that $f( x) =y$.

\

\begin{problem} d
neither a surjection nor an injection.
\end{problem}

$f( x) =x^{2}$ is neither an injection nor a surjection.

Not Injective: Choose some $a,b\in \mathbb{R}$ such that $a=-b$. We find that $f( a) =f( b) =a^{2}$ even though $a\neq b$.

Not Surjective: As an example, $-1\in \mathbb{R}$, but there is no $x$ such that $f( x) =-1$. The range of $f$ is $[ 0,\infty )$.

\
\hline
%%%%%%%%%%%%%%%%%%%%%%%%%%%%%%%%%%%%%%%%%%%%%%%%%%%%%%%%%%%%%%%%%%%%%%%%%%%%%%%%%
\section{}

\begin{problem*}
Let $f:A\to B$, and let $G_1,G_2\subseteq A$, and let $H_1,H_2\subseteq B$.
\end{problem*}

% (I'm assuming $f$ is an invertible function).

\begin{problem} a
Is it true that $f^{-1}(f(G_1))=G_1$? If so, prove it; if not, provide a counterexample, and provide the correct relation between the two sets, and justify your answer. 
\end{problem}

We know the image of $G_{1}$ under $f$, or $f( G_{1})$, will map to a set in $B$, which we'll call $G'$. The pre-image of $G'$, or $f^{-1}( G')$ will map back to the original values in $A$ that map to $G'$, which we originally said was the set $G_{1}$. Thus $f^{-1} (f(G_{1} ))=G_{1}$.

% Let $G'$ be the image of $G_{1}$ under $f$, or the set of all $f( x)$ such that $x\in G_{1}$. In other words, $G'=f( G_{1})$. Then the pre-image of $G'$ under $f$ is the set of all $x'$ such that $f( x') \in G'$. The pre-image effectively reverses the image, mapping back to the original domain of all $x\in G_{1}$. Thus the statement is true.

\

\begin{problem} b
Is it true that $f(f^{-1}(H_1))=H_1$? If so, prove it; if not, provide a counterexample, and provide the correct relation between the two sets, and justify your answer.
\end{problem}

The pre-image of $H_{1}$, or $f^{-1}( H_{1})$, will map to a set of values in $A$ (which we'll call $H'$) that are characterized by mapping to $H_{1}$ under $f$. The image of $H'$, or $f( H')$, will map to a set of values in $B$ under the relation $f$. But we originally said that this set was $H_{1}$, thus $f(f^{-1} (H_{1} ))=H_{1}$.

\

\begin{problem} c
Is it true that $f(G_1\cap G_2)=f(G_1)\cap f(G_2)$? If so, prove it; if not, provide a counterexample, and provide the correct relation between the two sets, and justify your answer.
\end{problem}

Not true: As a counterexample, let $f:[ 2]\rightarrow \{1\}$, and let $G_{1} =\{1\} ,\ G_{2} =\{2\}$. The left side evaluates to $f(G_{1} \cap G_{2} )=f( \varnothing) =\varnothing$. However, the right side evaluates to $f(G_{1} )\cap f(G_{2} )=\{1\} \cap \{1\} =\{1\}$.

The correct relation is $f(G_{1} \cap G_{2} )\subseteq f(G_{1} )\cap f(G_{2} )$. We know $G_{1} \cap G_{2} \subseteq G_{1}$, which implies $f( G_{1} \cap G_{2}) \subseteq f( G_{1})$. A similar argument but with $G_{2}$ shows $f( G_{1} \cap G_{2}) \subseteq f( G_{2})$. Combining these, we get $f(G_{1} \cap G_{2} )\subseteq f(G_{1} )\cap f(G_{2} )$.

A more standard way to prove this would be to choose some $x\in f (G_{1} \cap G_{2} )$. If we find the pre-image of both sides, we get $f^{-1}( x) \in G_{1} \cap G_{2}$. We can split this into $f^{-1}( x) \in G_{1}$ and $f^{-1}( x) \in G_{2}$, then find the images which are $x\in f( G_{1})$ and $x\in f( G_{2})$ respectively. Combining these yeilds $x\in f(G_{1} )\cap f(G_{2} )$. Thus $f(G_{1} \cap G_{2} )\subseteq f(G_{1} )\cap f (G_{2} )$.

\

\begin{problem} d
Is it true that $f^{-1}(H_1\cap H_2)=f^{-1}(H_1)\cap f^{-1}(H_2)$? If so, prove it; if not, provide a counterexample, and provide the correct relation between the two sets, and justify your answer.
\end{problem}

Choose some $h\in f^{-1} (H_{1} \cap H_{2} )$. If we find the image of both sides, we get $f( h) \in H_{1} \cap H_{2}$. We can split this into $f( h) \in H_{1}$ and $f( h) \in H_{2}$, then find the pre-images which are $h\in f^{-1}( H_{1})$ and $h\in f^{-1}( H_{2})$ respectively. Combining these yeilds $h\in f^{-1} (H_{1} )\cap f^{-1} (H_{2} )$. Thus $f^{-1} (H_{1} \cap H_{2} )\subseteq f^{-1} (H_{1} )\cap f^{-1} (H_{2} )$.



Choose some $h'\in f^{-1} (H_{1} )\cap f^{-1} (H_{2} )$. We can split this into $h'\in f^{-1}( H_{1})$ and $h'\in f^{-1}( H_{2})$. If we find the images of both sides of each, we get $f( h') \in H_{1}$ and $f( h') \in H_{2}$. Combining these yields $f( h') \in H_{1} \cap H_{2}$, and finding the pre-image of both sides gives $h'\in f^{-1}( H_{1} \cap H_{2})$. Thus $f^{-1} (H_{1} )\cap f^{-1} (H_{2} )\subseteq f^{-1}( H_{1} \cap H_{2})$.



Having proven subset equality in both directions, we can conclude $f^{-1} (H_{1} \cap H_{2} )=f^{-1} (H_{1} )\cap f^{-1} (H_{2} )$. 

\
\hline
%%%%%%%%%%%%%%%%%%%%%%%%%%%%%%%%%%%%%%%%%%%%%%%%%%%%%%%%%%%%%%%%%%%%%%%%%%%%%%%%%
\end{document}
%%%%%%%%%%%%%%%%%%%%%%%%%%%%%%%%%%%%%%%%%%%%%%%%%%%%%%%%%%%%%%%%%%%%%%%%%%%%%%%%%
\documentclass{article}
% Feel free to contact me for any reason:
%% Email 1: zain.kamal@rutgers.edu
%% Email 2: z.kamal2021@gmail.com
%% Discord: alci#6038

% Last updated 1/28/22

% Template based off of Justin Kim's (don't know where he got it from originally), but I've made a ton of edits. His soul lives on in random packages and commands I'm too lazy to comment out.


%%%%%%%%%%%%%%%%%%%%%%%%%%%%%%%%%%%%%%%%%%%%%%%%%%%%%%%%%%%%%%%%%%%%%%%%%%%%%%%%%%%%
%%% Default packages:

\usepackage[margin=1in]{geometry} 
\usepackage{amsmath,amsthm,amssymb,amsfonts, fancyhdr, color, comment, graphicx, environ}
\usepackage{xcolor}
\usepackage{mdframed}
\usepackage{bm}
\usepackage[shortlabels]{enumitem}
\usepackage{mathtools}
\usepackage{listings}
\usepackage{stmaryrd}
\usepackage{indentfirst}
\usepackage{hyperref}


%%%%%%%%%%%%%%%%%%%%%%%%%%%%%%%%%%%%%%%%%
%%% Pacakages I've added:

%% For Math 300 (Winter 2021):

\usepackage{mathdots} % for \iddots

\usepackage[ruled]{algorithm2e} % Algorithms
% NOTE: FIND A BETTER ALGORITHM PACKAGE, OR ATLEAST LEARN HOW TO USE THIS ONE BECAUSE DOUBLE INDENTING IS A FUCKING NIGHTMARE (or just import from mathcha?)
%% Example algorithm:
% \begin{center}
% 	\begin{minipage}{0.5\linewidth} % Adjust the minipage width to accomodate for the length of algorithm lines
% 		\begin{algorithm}[H]
% 			\KwIn{$(a, b)$, two floating-point numbers}  % Algorithm inputs
% 			\KwResult{$(c, d)$, such that $a+b = c + d$} % Algorithm outputs/results
% 			\medskip
% 			\If{$\vert b\vert > \vert a\vert$}{
% 				exchange $a$ and $b$ \;
% 			}
% 			$c \leftarrow a + b$ \;
% 			$z \leftarrow c - a$ \;
% 			$d \leftarrow b - z$ \;
% 			{\bf return} $(c,d)$ \;
% 			\caption{\texttt{FastTwoSum}} % Algorithm name
% 			\label{alg:fastTwoSum}   % optional label to refer to
% 		\end{algorithm}
% 	\end{minipage}
% \end{center}



%%%%%%%%%%%%%%%%%%%%%%%%%%%%%%%%%%%%%%%%%%%%%%%%%%%%%%%%%%%%%%%%%%%%%%%%%%%%%%%%%%%%
%%% Default commands:

\renewcommand{\vec}[1]{\mathbf{#1}}
	
\newcommand{\WidestEntry}{$lon_1$}%
\newcommand{\SetToWidest}[1]{\makebox[\widthof{\WidestEntry}]{#1}}%
\newcommand\tab[1][0.61cm]{\hspace*{#1}}
\newcommand{\nats}{\mathbb{N}}
\newcommand{\rats}{\mathbb{Q}}
\newcommand{\reals}{\mathbb{R}}
\newcommand{\Z}[1]{\mathbb{Z}_{#1}}
\newcommand{\BigO}[1]{\mathcal{O}(#1)}
\newcommand{\seq[1]}{(#1_n)}
\newcommand{\subseq[1]}{(#1_{n_k})}
\newcommand{\Lim}[2]{\lim \limits _{#1 \to #2}}
\newcommand{\Min}[2]{\min \{#1, #2\}}
\newcommand{\inv}{^{-1}}
\newcommand{\h}{^\text{th}}
\newcommand{\lrangle}[1]{\langle #1 \rangle}
\newcommand{\abs}[1]{\left\lvert #1 \right\rvert}

\DeclarePairedDelimiter{\ceil}{\lceil}{\rceil}
\DeclarePairedDelimiter{\floor}{\lfloor}{\rfloor}
\DeclareMathOperator{\supp}{supp}
\DeclareMathOperator{\rad}{rad}
\DeclareMathOperator*{\argmin}{arg\,min}
\DeclareMathOperator*{\argmax}{arg\,max}
\DeclareMathOperator*{\Var}{Var}
\DeclareMathOperator*{\Cov}{Cov}
\DeclareMathOperator*{\Corr}{Corr}
\DeclareMathOperator*{\Aut}{Aut}
\newcommand{\prob}[1]{\section*{Problem #1}}


%%%%%%%%%%%%%%%%%%%%%%%%%%%%%%%%%%%%%%%%%
%%% Commands I've added:

%% For Math 300 (Winter 2021):

\newcommand{\lrbrace}[1]{\{ #1 \}}
\newcommand{\powerset}{\mathcal{P}}
\newcommand{\ints}{\mathbb{Z}}

% Source/inspiration: https://tex.stackexchange.com/a/42728:
\newcommand{\numberthis}{\addtocounter{equation}{1}\tag{\theequation}\label{\theequation}}
    % Within an `align*` environment, put `\numberthis` after a line to number it. 
    % Access it with `\eqref{ [number of equation] }`
\newcommand{\numberthiswith}[1]{\addtocounter{equation}{1}\tag{\theequation}\label{#1}}
    % Within an `align*` environment, put `\numberthiswith{ [your_label] }` after a line to number it. 
    % Access it with `\eqref{ [your_label] }`



%%%%%%%%%%%%%%%%%%%%%%%%%%%%%%%%%%%%%%%%%%%%%%%%%%%%%%%%%%%%%%%%%%%%%%%%%%%%%%%%%%%%
%%% Default formatting:

\hypersetup{
    colorlinks=true,
    linkcolor=blue,
    filecolor=magenta,      
    urlcolor=blue,
}

\setlength{\parindent}{0cm}
\setlength{\parskip}{6pt}

\pagestyle{fancy}


%% Misc formatting additions

% make bullets with itemize much smaller
\newlength{\mylen}
\setbox1=\hbox{$\bullet$}\setbox2=\hbox{\tiny$\bullet$}
\setlength{\mylen}{\dimexpr0.5\ht1-0.5\ht2}
\renewcommand\labelitemi{\raisebox{\mylen}{\tiny$\bullet$}}


% Modified version of problem environment below
% \newenvironment{problem}[2][Problem]
%     { \begin{mdframed}[backgroundcolor=gray!5] \textbf{#1 #2} \\}
%     {  \end{mdframed}}
% \newenvironment{solution}{\textbf{Solution}\\}


%%%%%%%%%%%%%%%%%%%%%%%%%%%%%%%%%%%%%%%%%
%%% Formatting I've added:

%% Grey boxes for problem statements (note that I don't have a "solution" section):

% Problem environment, but shows "(a)" instead of "Problem a"
\newenvironment{problem}[2][]
    { \begin{mdframed}[backgroundcolor=gray!5] \textbf{#1 (#2)}}
    {  \end{mdframed}}
% Problem environment, but no "([input_char])" at all
\newenvironment{problem*}
    { \begin{mdframed}[backgroundcolor=gray!5] \\}
    {  \end{mdframed}}


% Example environment, currently identical to "problem" (Note: this is better written than the problem environment because I wrote it myself from scratch. Use this as an example for future new environments.)
\newcounter{example}[section]
\newenvironment{example}
    { 
        \refstepcounter{example}
        \begin{mdframed}[backgroundcolor=gray!5]
        \textbf{\\Example \thesection.\theexample:}
    }
    {\\ \end{mdframed}}


%%%%%%%%%%%%%%%%%%%%%%%%%%%%%%%%%%%%%%%%%%%%%%%%%%%%%%%%%%%%%%%%%%%%%%%%%%%%%%%%%%%%
% Misc things I've added




%%%%%%%%%%%%%%%%%%%%%%%%%%%%%%%%%%%%%%%%%%%%%
% Fill in the appropriate information below
\lhead{Zain Kamal}
% \rhead{Math 244 Spring 2022} % Moved to document
% \chead{\textbf{Homework 2}} % Moved to document
\begin{document}
\chead{\textbf{Homework 2}}
%%%%%%%%%%%%%%%%%%%%%%%%%%%%%%%%%%%%%%%%%%%%%%%%%%%%%%%%%%%%%%%%%%%%%%%%%%%%%%%%%
\section{}


\begin{problem} a
What set (that we know) is $\bigcup _{x \in \mathbb{Z}} \{x, x + 1, x + 2\}$?
\end{problem}

This is equal to the set of all integers, $\mathbb{Z}$. 

This is because $\bigcup \limits_{x \in \mathbb{Z}} \{x\} = 
\bigcup \limits_{x \in \mathbb{Z}} \{x + 1\} =
\bigcup \limits_{x \in \mathbb{Z}} \{x + 2\} =
\mathbb{Z}$. Redundancies are removed in sets.


\


\begin{problem} b
What set (that we know) is $\bigcup _{n \in \nat} (-n, n)$?
\end{problem}

This is equal to the set of all real numbers, $\reals$.


\

\begin{problem} c
What set is $\bigcap _{n \in \nat} (-n, n)$? Use formal set builder notation to describe it.
\end{problem}

$\{ x \in \reals \mid (x > -1) \land (x < 1) \} = (-1,1)$.

\

\begin{problem} d
What set is $\bigcup_{n = 2} ^\infty [0, 1 - 1/n)$ \verb+(\bigcup_{n = 2} ^\infty [0, 1 - 1/n))+?
\end{problem}

$\{ x \in \reals \mid ( x \geq 0 ) \land (x < 1) \} = [0,1)$.


\

\begin{problem} e
What set is $\bigcup_{x \in \mathbb{Z}} \left( \bigcup_{n = 2} ^\infty [x, x + 1 - 1/n) \right)$? Express your answer as simply as possible.
\end{problem}

This is equal to $\reals$.

The problem simplifies to $\bigcup \limits_{x \in \mathbb{Z}} \{[x, x+1)\} = (-\infty, \infty)$.


\
\hline
%%%%%%%%%%%%%%%%%%%%%%%%%%%%%%%%%%%%%%%%%%%%%%%%%%%%%%%%%%%%%%%%%%%%%%%%%%%%%%%%%
\section{} 

\begin{problem*}
For the sets in problem 1, find the complement of each set with respect to $\mathbb{R}$. (Recall that the complement of $A$ with respect to $B$ is every element in $B$ that is not in $A$)
\end{problem*}

\begin{problem} a
$\mathbb{Z}^C$.
\end{problem}

$\{ x \in \reals \mid x \notin \mathbb{Z} \} = \bigcup \limits_{x \in \mathbb{Z}} (x, x+1)$.

\textit{Question from student: of the two above, which method is better?}

\

\begin{problem} b
$\reals ^C$
\end{problem}

$\varnothing$.

\begin{problem} c
$(-1,1)^C$
\end{problem}

$\{ x \in \reals \mid (x \leq -1) \lor (x \geq 1) \} = (-\infty, -1] \cup [1, \infty)$.

\

\begin{problem} d
$[0,1)^C$.
\end{problem}

$\{ x \in \reals \mid (x < 0) \lor (x \geq 1) \} = (-\infty, 0) \cup [1, \infty)$.

\

\begin{problem} e
$\reals ^C$.
\end{problem}

$\varnothing$.

\
\hline
%%%%%%%%%%%%%%%%%%%%%%%%%%%%%%%%%%%%%%%%%%%%%%%%%%%%%%%%%%%%%%%%%%%%%%%%%%%%%%%%%
\section{}

\begin{problem} a
Prove that the cardinality of $|A \times B|$ is $|A| \cdot |B|$, assuming that $A$ and $B$ are finite. It does not have to be particularly rigorous.
\end{problem}

The definition of the cartesian product is $A \times B = \{ (a,b) \mid a \in A, b \in B \}$. We have $|A|$ unique options for $a$, and for each of those $a$, we have $|B|$ unique options for $b$. Therefore, there are $|A| \cdot |B|$ unique elements in $A \times B$ — in other words, $|A \times B| = |A| \cdot |B|$.

\

\begin{problem} b
What is the cardinality of $\mathcal{P}(\mathcal{P}(\mathcal{P}(\mathcal{P}(\varnothing))))$? What about if I apply the powerset function to the empty set $n$ times in general?
\end{problem}

$|\mathcal{P}(\mathcal{P}(\mathcal{P}(\mathcal{P}(\varnothing))))| = 16$.

Recall that for an arbitrary finite set $A$, $|\mathcal{P}(A)| = 2^{(|A|)}$. If we were to apply the powerset function again, we'd get $| \powerset ( \powerset (A) ) | = 2^{(| \powerset (A) |)} = 2^{(2^{(|A|)})}$. In our case, $A = \varnothing$, so $|A|=0$, which makes $2^{(2^{(|A|)})}=2$.


Generalizing this, applying the powerset function $n$ times to the empty set yields a set with cardinality $2^{\iddots ^ 2}$, where $2$ appears $n-1$ times (or in ``\href{https://en.wikipedia.org/wiki/Tetration}{tetration}'' notation, $^{(n-1)} 2$).

This is clear by manually calculating each nested powerset:

\begin{itemize}
    \item $|\varnothing| = 0$,
    \item $|\powerset (\varnothing)| = 2^0 = 1$,
    \item $|\powerset ( \powerset ( \varnothing) )| = 2^{2^0} = 2$,
    \item $|\powerset ( \powerset ( \powerset ( \varnothing) ) )| = 2^{2^{2^0}} = 4$,
    \item $|\powerset ( \powerset ( \powerset ( \powerset ( \varnothing) ) ) )| = 2^{2^{2^{2^0}}} = 16$.
\end{itemize}

\
\hline
%%%%%%%%%%%%%%%%%%%%%%%%%%%%%%%%%%%%%%%%%%%%%%%%%%%%%%%%%%%%%%%%%%%%%%%%%%%%%%%%%
\section{}

\begin{problem} a
Prove that $(A \setminus B) \subseteq A$.
\end{problem}

% By Definition 2.17, $A \setminus B := \lrbrace{a \in A \mid a \notin B}$. Thus, it follows that 

% Let $x \in (A \setminus B)$. By the definition of the set difference, we know that $\lrbrace{x \in A \mid x \notin B}$. 

By Definition 2.17, $A \setminus B := \lrbrace{x \in A \mid x \notin B}$. Thus it trivially follows that $x$ is an element of $A$. Because an arbitrary element of $A \setminus B$ is necessarily an element of $A$, we can conclude $A \setminus B \subseteq A$. 

% In the case $B = \varnothing$, $x$ will never be an element of $B$. Therefore, our equation for the set difference simplifies to $A \setminus B := \lrbrace{x \in A} = A$.

% Because $A \setminus B \subset A$ and it is possible for $A \setminus B =A$, we can conclude $A \setminus B \subseteq A$.

\

\begin{problem} b
Prove the first DeMorgan's law for sets that I listed above: \[(A\cup B)^C = A^C \cap B^C.\]
\end{problem}

Let $x \in (A \cup B)^C$. 

Then $x \notin (A \cup B)$,

$(x \notin A) \land (x \notin B)$,

$(x \in A^C) \land (x \in B^C)$,

$x \in (A^C \cap B^C)$.

Therefore, $(A \cup B)^C \subseteq (A^C \cap B^C)$.

\

Let $y \in (A^C \cap B^C)$.

Then $(y \in A^C) \land (y \in B^C)$,

$(y \notin A) \land (y \notin B)$,

$y \notin (A \cup B)$,

$y \in (A \cup B)^C$.

Therefore $(A^C \cap B^C) \subseteq (A\cup B)^C$.

\

$(A \cup B)^C \subseteq (A^C \cap B^C)$ and $(A^C \cap B^C) \subseteq (A\cup B)^C$, so $(A\cup B)^C = A^C \cap B^C$.

\
\hline
%%%%%%%%%%%%%%%%%%%%%%%%%%%%%%%%%%%%%%%%%%%%%%%%%%%%%%%%%%%%%%%%%%%%%%%%%%%%%%%%%
\section{}

\begin{problem*}
Define the \textbf{symmetric difference} of two sets $A,B$ as follows: \[A\triangle B:= (A\setminus B)\cup (B\setminus A).\] Prove that $A\triangle B=(A\cup B)\setminus (A\cap B)$. 
\end{problem*}

\begin{itemize}
    \item We begin with the definition of the symmetric difference: \\ 
    $(A \setminus B) \cup (B \setminus A)$.
    \item By the definition of set minus: \\
    $(A \cap B^C) \cup (B \cap A^C)$.
    \item By distribution [of form $P \cup (Q \cap R)$, where $P=(A \cap B^C)$, $Q=B$, and $R=A^C$]: \\
    $((A \cap B^C) \cup B) \cap ((A \cap B^C) \cup A^C)$.
    \item By distribution [of form $(P\cap Q) \cup R$]: \\
    $((A \cup B)\cap(B^C \cup B))\cap((A \cup A^C)\cap(B^C \cup A^C))$.
    \item By law of excluded middle and definition of truth: \\
    $(A \cup B) \cap (B^C \cup A^C)$.
    \item By commutivity of union: \\
    $(A \cup B) \cap (A^C \cup B^C)$.
    \item By definition of set minus: \\
    $(A \cup B) \setminus (A^C \cup B^C)^C$.
    \item By DeMorgan's first law for sets: \\
    $(A \cup B) \setminus (A \cap B)$.
\end{itemize}

Therefore, we find that $A\triangle B = (A\setminus B)\cup (B\setminus A) = (A \cup B) \setminus (A \cap B)$. \quad \qedsymbol

\
\hline
%%%%%%%%%%%%%%%%%%%%%%%%%%%%%%%%%%%%%%%%%%%%%%%%%%%%%%%%%%%%%%%%%%%%%%%%%%%%%%%%%
\section{}
\begin{problem*}
We can perform unions and intersections without an indexing set, as well. For example, we can write \[\bigcup_{X \in \mathcal{P}(A)} X = A.\] (You can prove this to yourself if you'd like). Now let's define the following set: \[\mathcal{T}_A := \{X \subseteq A \mid |X| = 2\}.\] We say this set is \textit{parameterized} by a set $A$. Now, for which $A$ is the union \[\bigcup_{X \in \mathcal{T}_A} X = A?\]
\end{problem*}

$\bigcup \limits_{X \in \mathcal{T}_A} X = A$ for all $A$ with a cardinality greater than or equal to 2.





%%%%%%%%%%%%%%%%%%%%%%%%%%%%%%%%%%%%%%%%%%%%%%%%%%%%%%%%%%%%%%%%%%%%%%%%%%%%%%%%%
\end{document}
%%%%%%%%%%%%%%%%%%%%%%%%%%%%%%%%%%%%%%%%%%%%%%%%%%%%%%%%%%%%%%%%%%%%%%%%%%%%%%%%%
\documentclass{article}
% Feel free to contact me for any reason:
%% Email 1: zain.kamal@rutgers.edu
%% Email 2: z.kamal2021@gmail.com
%% Discord: alci#6038

% Last updated 1/28/22

% Template based off of Justin Kim's (don't know where he got it from originally), but I've made a ton of edits. His soul lives on in random packages and commands I'm too lazy to comment out.


%%%%%%%%%%%%%%%%%%%%%%%%%%%%%%%%%%%%%%%%%%%%%%%%%%%%%%%%%%%%%%%%%%%%%%%%%%%%%%%%%%%%
%%% Default packages:

\usepackage[margin=1in]{geometry} 
\usepackage{amsmath,amsthm,amssymb,amsfonts, fancyhdr, color, comment, graphicx, environ}
\usepackage{xcolor}
\usepackage{mdframed}
\usepackage{bm}
\usepackage[shortlabels]{enumitem}
\usepackage{mathtools}
\usepackage{listings}
\usepackage{stmaryrd}
\usepackage{indentfirst}
\usepackage{hyperref}


%%%%%%%%%%%%%%%%%%%%%%%%%%%%%%%%%%%%%%%%%
%%% Pacakages I've added:

%% For Math 300 (Winter 2021):

\usepackage{mathdots} % for \iddots

\usepackage[ruled]{algorithm2e} % Algorithms
% NOTE: FIND A BETTER ALGORITHM PACKAGE, OR ATLEAST LEARN HOW TO USE THIS ONE BECAUSE DOUBLE INDENTING IS A FUCKING NIGHTMARE (or just import from mathcha?)
%% Example algorithm:
% \begin{center}
% 	\begin{minipage}{0.5\linewidth} % Adjust the minipage width to accomodate for the length of algorithm lines
% 		\begin{algorithm}[H]
% 			\KwIn{$(a, b)$, two floating-point numbers}  % Algorithm inputs
% 			\KwResult{$(c, d)$, such that $a+b = c + d$} % Algorithm outputs/results
% 			\medskip
% 			\If{$\vert b\vert > \vert a\vert$}{
% 				exchange $a$ and $b$ \;
% 			}
% 			$c \leftarrow a + b$ \;
% 			$z \leftarrow c - a$ \;
% 			$d \leftarrow b - z$ \;
% 			{\bf return} $(c,d)$ \;
% 			\caption{\texttt{FastTwoSum}} % Algorithm name
% 			\label{alg:fastTwoSum}   % optional label to refer to
% 		\end{algorithm}
% 	\end{minipage}
% \end{center}



%%%%%%%%%%%%%%%%%%%%%%%%%%%%%%%%%%%%%%%%%%%%%%%%%%%%%%%%%%%%%%%%%%%%%%%%%%%%%%%%%%%%
%%% Default commands:

\renewcommand{\vec}[1]{\mathbf{#1}}
	
\newcommand{\WidestEntry}{$lon_1$}%
\newcommand{\SetToWidest}[1]{\makebox[\widthof{\WidestEntry}]{#1}}%
\newcommand\tab[1][0.61cm]{\hspace*{#1}}
\newcommand{\nats}{\mathbb{N}}
\newcommand{\rats}{\mathbb{Q}}
\newcommand{\reals}{\mathbb{R}}
\newcommand{\Z}[1]{\mathbb{Z}_{#1}}
\newcommand{\BigO}[1]{\mathcal{O}(#1)}
\newcommand{\seq[1]}{(#1_n)}
\newcommand{\subseq[1]}{(#1_{n_k})}
\newcommand{\Lim}[2]{\lim \limits _{#1 \to #2}}
\newcommand{\Min}[2]{\min \{#1, #2\}}
\newcommand{\inv}{^{-1}}
\newcommand{\h}{^\text{th}}
\newcommand{\lrangle}[1]{\langle #1 \rangle}
\newcommand{\abs}[1]{\left\lvert #1 \right\rvert}

\DeclarePairedDelimiter{\ceil}{\lceil}{\rceil}
\DeclarePairedDelimiter{\floor}{\lfloor}{\rfloor}
\DeclareMathOperator{\supp}{supp}
\DeclareMathOperator{\rad}{rad}
\DeclareMathOperator*{\argmin}{arg\,min}
\DeclareMathOperator*{\argmax}{arg\,max}
\DeclareMathOperator*{\Var}{Var}
\DeclareMathOperator*{\Cov}{Cov}
\DeclareMathOperator*{\Corr}{Corr}
\DeclareMathOperator*{\Aut}{Aut}
\newcommand{\prob}[1]{\section*{Problem #1}}


%%%%%%%%%%%%%%%%%%%%%%%%%%%%%%%%%%%%%%%%%
%%% Commands I've added:

%% For Math 300 (Winter 2021):

\newcommand{\lrbrace}[1]{\{ #1 \}}
\newcommand{\powerset}{\mathcal{P}}
\newcommand{\ints}{\mathbb{Z}}

% Source/inspiration: https://tex.stackexchange.com/a/42728:
\newcommand{\numberthis}{\addtocounter{equation}{1}\tag{\theequation}\label{\theequation}}
    % Within an `align*` environment, put `\numberthis` after a line to number it. 
    % Access it with `\eqref{ [number of equation] }`
\newcommand{\numberthiswith}[1]{\addtocounter{equation}{1}\tag{\theequation}\label{#1}}
    % Within an `align*` environment, put `\numberthiswith{ [your_label] }` after a line to number it. 
    % Access it with `\eqref{ [your_label] }`



%%%%%%%%%%%%%%%%%%%%%%%%%%%%%%%%%%%%%%%%%%%%%%%%%%%%%%%%%%%%%%%%%%%%%%%%%%%%%%%%%%%%
%%% Default formatting:

\hypersetup{
    colorlinks=true,
    linkcolor=blue,
    filecolor=magenta,      
    urlcolor=blue,
}

\setlength{\parindent}{0cm}
\setlength{\parskip}{6pt}

\pagestyle{fancy}


%% Misc formatting additions

% make bullets with itemize much smaller
\newlength{\mylen}
\setbox1=\hbox{$\bullet$}\setbox2=\hbox{\tiny$\bullet$}
\setlength{\mylen}{\dimexpr0.5\ht1-0.5\ht2}
\renewcommand\labelitemi{\raisebox{\mylen}{\tiny$\bullet$}}


% Modified version of problem environment below
% \newenvironment{problem}[2][Problem]
%     { \begin{mdframed}[backgroundcolor=gray!5] \textbf{#1 #2} \\}
%     {  \end{mdframed}}
% \newenvironment{solution}{\textbf{Solution}\\}


%%%%%%%%%%%%%%%%%%%%%%%%%%%%%%%%%%%%%%%%%
%%% Formatting I've added:

%% Grey boxes for problem statements (note that I don't have a "solution" section):

% Problem environment, but shows "(a)" instead of "Problem a"
\newenvironment{problem}[2][]
    { \begin{mdframed}[backgroundcolor=gray!5] \textbf{#1 (#2)}}
    {  \end{mdframed}}
% Problem environment, but no "([input_char])" at all
\newenvironment{problem*}
    { \begin{mdframed}[backgroundcolor=gray!5] \\}
    {  \end{mdframed}}


% Example environment, currently identical to "problem" (Note: this is better written than the problem environment because I wrote it myself from scratch. Use this as an example for future new environments.)
\newcounter{example}[section]
\newenvironment{example}
    { 
        \refstepcounter{example}
        \begin{mdframed}[backgroundcolor=gray!5]
        \textbf{\\Example \thesection.\theexample:}
    }
    {\\ \end{mdframed}}


%%%%%%%%%%%%%%%%%%%%%%%%%%%%%%%%%%%%%%%%%%%%%%%%%%%%%%%%%%%%%%%%%%%%%%%%%%%%%%%%%%%%
% Misc things I've added




%%%%%%%%%%%%%%%%%%%%%%%%%%%%%%%%%%%%%%%%%%%%%
% Fill in the appropriate information below
\lhead{Zain Kamal}
% \rhead{Math 244 Spring 2022} % Moved to document
% \chead{\textbf{Homework 2}} % Moved to document
\begin{document}
\chead{\textbf{Homeworm 5}}
%%%%%%%%%%%%%%%%%%%%%%%%%%%%%%%%%%%%%%%%%%%%%%%%%%%%%%%%%%%%%%%%%%%%%%%%%%%%%%%%%
\section{}

\begin{problem*}
Find the error in the proof of Proposition 5.14 (it is not true!). Respond in the homework channel on Discord rather than here, and discuss potential solutions.
\end{problem*}

The proof wrongfully assumes that $s$ exists. This is not the case in a tree with a single vertex ($G=(\lrbrace{v_1},\lrbrace{})$), where $v_1$ is not connected to any vertices.

This can be fixed by specifying that $V$ has atleast 2 elements ($|V|\geq 2$) at the beginning of the proof.


\
\hline
%%%%%%%%%%%%%%%%%%%%%%%%%%%%%%%%%%%%%%%%%%%%%%%%%%%%%%%%%%%%%%%%%%%%%%%%%%%%%%%%%
\section{}

\begin{problem*}
Prove that the sum of degrees of vertices in a simple graph is even.
\end{problem*}

Let $P( n)$ be the statement that the sum of degrees of vertices in a simple graph with $n$ vertices is even. 

$P( 0)$ is true, as a graph with 0 vertices has 0 connections. $P( 1)$ is also true, as a graph with 1 vertex has 0 connections. 

\

Now suppose $P( k)$ is true for some integer $k\geq 1$. Let $G$ be a simple graph with $k+1$ vertices. Choose some vertex $v$ in $G$, and let $V'$ be the set of vertices connected to $v$. Removing $v$ from $G$ will decrease the sum of degrees of vertices by $2\cdot |V'|$, as $v$ has a degree of $|V'|$, and the degree of all vertices in $V'$ will decrease by one. 

Note that $2\cdot |V'|$ is even because $2\cdot |V'|=2k$' for some integer $k'=|V'|\in \ints$.

$G$ is now a simple graph with $k$ vertices, so the sum of their degrees is even. If we re-add $v$ (and its original connections) to $G$, the sum of degrees of vertices will increase by $2\cdot |V'|$. The sum of two even numbers is even, so $P( k+1)$ follows. Thus by induction, $P( n)$ is true for all integers $n\geq 0$. \qed


\
\hline
%%%%%%%%%%%%%%%%%%%%%%%%%%%%%%%%%%%%%%%%%%%%%%%%%%%%%%%%%%%%%%%%%%%%%%%%%%%%%%%%%
\section{}

\begin{problem} a
Prove that a connected, simple graph with $|V| - 1$ edges is acyclic.
\end{problem}

Let $P( n)$ be the statement that at least $n-1$ edges are required to connect a simple graph with $n$ vertices. $P( 1)$ is true, as a simple graph with one vertex has zero edges and is connected.

Now suppose $P( k)$ for some natural number $k\geq 1$. Take a simple graph with $k+1$ vertices and zero edges. Removing an arbitrary vertex, $v$, allows us to form a connected graph with $k-1$ edges, which is the minimum. If we now re-add $v$, we need to add a minimum of one edge to ensure connectedness. Thus we have a graph with $k+1$ vertices and the minimum number of edges, $( k-1) +1=k$, so $P( k+1)$ follows. Thus by induction, $P( n)$ is true for all $n\in \mathbb{N}$.

\

Now suppose for the sake of contradiction that there exists a connected simple graph with $|V|-1$ edges that is also cyclic. Because it's cyclic, we can remove at least one edge while remaining connected. However, this leads to a connected graph with less than $|V|-1$ edges, which contradicts $P( |V|)$. Therefore, a connected simple graph with $|V|-1$ edges must be acyclic.\qed

\

\begin{problem} b
Prove that an acyclic, simple graph with $|V| - 1$ edges is connected.
\end{problem}

Let $P( n)$ be the statement that an acyclic, simple graph with $n$ vertices and $n-1$ edges is connected. $P( 1)$ is true because there is only one configuration (a single vertex), and it is both acyclic and connected.

\

Now supposed $P( k)$ for some natural number $k\geq 1$. Let $G$ be a simple, acyclic graph with $k+1$ vertices and $k$ edges. Because $G$ is acyclic and has atleast one edge, we can find a subgraph of $G$ containing a set of connected, acyclic vertices. This subgraph is a tree, so it must have at least two leaves by Proposition 5.14 (note that we escape the error in problem 1 because there are at least two vertices). Thus $G$ has at least two leaves. Removing one of these leaves will remove a single edge, resulting in a new graph, $G'$, with $k$ vertices and $k-1$ edges. $G'$ is still acyclic as we are not adding any new edges, so $P( k)$ implies that $G'$ is connected. If we re-add the leaf (and edge) we removed, we have a simple, acyclic graph with $k+1$ vertices and $k$ edges that is connected, so $P( k+1)$ follows. Thus by induction, $P( n)$ is true for all $n\in \mathbb{N}$.\qed

\
\hline
%%%%%%%%%%%%%%%%%%%%%%%%%%%%%%%%%%%%%%%%%%%%%%%%%%%%%%%%%%%%%%%%%%%%%%%%%%%%%%%%%
\section{}

\begin{problem*}
Prove that a simple graph is bipartite iff it contains no odd-length cycles.
\end{problem*}

($\Rightarrow $) Divide the vertices of a bipartite graph into two disjoint and independent sets, $U$ and $V$ such that every edge connects a vertex in $U$ to one in $V$. Suppose for the sake of contradiction that the bipartite graph contains an odd-length cycle, represented by the path $v_{1} ,v_{2} ,\dotsc ,v_{n} ,v_{1}$ where $n$ must be odd. Let $v_{1}$ be an element of one of the sets, say $U$. Then $v_{i} \in U$ if $i$ is odd (there exists some integer $k$ such that $i=2k+1$) as every 2 "steps" bring us back to an element of $U$. Thus $v_{n} \in U$, but this leads to a contradiction as $v_{n}$ and $v_{1}$ share an edge but are both elements of $U$. Therefore if a graph is bipartite, then it contains no odd-length cycles.

\

($\Leftarrow $) Let $G$ be a connected graph with no odd-length cycles. Choose some vertex $v$. Let $U$ be the set of vertices with an odd-length path from $v$, and let $V$ be the set of vertices with an even-length path from $v$. Thus $U$ and $V$ are disjoint, and together they comprise all vertices in $G$.

Suppose for the sake of contradiction that there exist two unique vertices, $v_{a}$ and $v_{b}$, such that they belong to the same set ($v_{a} ,v_{b} \in U$ or $v_{a} ,v_{b} \in V$) and are connected by an edge. 
\begin{itemize}
\item Case 1: If $v_{a} =v$, then the distance between $v$ and $v_{a}$ is zero, which is even. Vertex $v_{b}$ is in the same set as $v_{a}$, so the distance between $v_{b}$ and $v$ (equivalently, the distance between $v_{b}$ and $v_{a}$) must also be even. But this leads to a contradicion, as $v_{a}$ and $v_{b}$ being connected by an edge would imply the distance between them is one, which is odd.
\item Case 2: Assume $v_{a} \neq v_{b} \neq v$. Let $A$ be the shortest path from $v$ to $v_{a}$, and let $B$ be the shortest path from $v$ to $v_{b}$. Vertices $v_{a}$ and $v_{b}$ are both in the same set, so the lengths of $A$ and $B$ must be of the same parity. Thus, traveling from $v_{a}$ to $v_{b}$ by traversing path $A$ in the reverse direction and then path $B$ in the forward direction will have an even-length, as the sum of two numbers with the same parity is even. This means there exists some integer $k$ such that the path length equals $2k$. If we then travel from $v_{b}$ to $v_{a}$ via the edge that connects them, the path length becomes $2k+1$. However this leads to a contradiction as we now have an odd-length cycle.
\end{itemize}

Thus, every edge connects a vertex in $U$ to one in $V$ and vice versa, so $G$ is bipartite.\qed

\
\hline
%%%%%%%%%%%%%%%%%%%%%%%%%%%%%%%%%%%%%%%%%%%%%%%%%%%%%%%%%%%%%%%%%%%%%%%%%%%%%%%%%
\section{}

\begin{problem*}
A simple graph is said to be $k$-colorable if there is a function $f:V \to [k]$ such that for any edge $(a, b)$, $f(a) \neq f(b)$.
\end{problem*}

\begin{problem} a
Show that a simple graph is 2-colorable iff it is bipartite.
\end{problem}

($ \Rightarrow $) Let $ G$ be a 2-colorable graph, meaning all vertices get one of two colors and no vertices that share an edge have the same color. Let $ U$ be the set of vertices with the first color, and let $ V$ be the set of vertices with the second color. No vertices are in both sets, both sets comprise all vertices, and every edge connects vertices of different colors. Thus $ G$ is bipartite.

\

($ \Leftarrow $) Let $ G$ be a bipartite graph. Divide the vertices into two disjoint and independent sets, $U$ and $V$, such that every edge connects a vertex in $U$ to one in $V$. Coloring the vertices in $ U$ the first color and $ V$ the second color ensures vertices that share an edge have different colors, and thus is a valid 2-coloring.\qed

\

\begin{problem} b
$\dagger$ Show that a simple graph whose vertices all have degree at most $k$ is $(k+1)$-colorable.
\end{problem}

Let $ P( n)$ be the statement that a simple graph whose vertices all have degree at most $ n$ is $(n+1)$-colorable. $ P( 1)$ is true, as a single vertex must have degree zero and is 1-colorable.

\

Suppose $ P( k)$ is true for some natural number $ k\geq 1$. Take a simple graph with $ k+2$ vertices and degree at most $ k+1$ (eg star graph $ S_{k+1})$. If we remove one vertex, it is still a simple graph but the degree is now at most $ k$, thus $ P( k)$ implies it is $ ( k+1)$-colorable. We can now re-add the vertex (and edges) back to the graph and color the vertex differently from all other verticies to ensure a valid coloring. This yields a simple graph with degree at most $ k+1$ that is $ ( k+2)$-colorable, so $ P( k+1)$ follows. Thus by induction, $P(n)$ holds for all $n \in \nats$. \qed 

\
\hline
%%%%%%%%%%%%%%%%%%%%%%%%%%%%%%%%%%%%%%%%%%%%%%%%%%%%%%%%%%%%%%%%%%%%%%%%%%%%%%%%%
\end{document}
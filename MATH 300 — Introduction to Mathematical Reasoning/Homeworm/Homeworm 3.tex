%%%%%%%%%%%%%%%%%%%%%%%%%%%%%%%%%%%%%%%%%%%%%%%%%%%%%%%%%%%%%%%%%%%%%%%%%%%%%%%%%
\documentclass{article}
% CREDITS TO JUSTIN KIM (idk where he got this from)

%%%%%%%%%%%%%%%%%%%%%%%%%%%%%%%%%%%%%%%%%%%%%%%%%%%%%%%%%%%%%%%%%%%%%%%%%%%%%%%%%%%%
%%% Default packages:

\usepackage[margin=1in]{geometry} 
\usepackage{amsmath,amsthm,amssymb,amsfonts, fancyhdr, color, comment, graphicx, environ}
\usepackage{xcolor}
\usepackage{mdframed}
\usepackage{bm}
\usepackage[shortlabels]{enumitem}
\usepackage{mathtools}
\usepackage{listings}
\usepackage{stmaryrd}
\usepackage{indentfirst}
\usepackage{hyperref}


%%%%%%%%%%%%%%%%%%%%%%%%%%%%%%%%%%%%%%%%%
%%% Pacakages I've added:

%% For Math 300 (Winter 2021):

\usepackage{mathdots} % for \iddots

\usepackage[ruled]{algorithm2e} % Algorithms
% NOTE: FIND A BETTER ALGORITHM PACKAGE, OR ATLEAST LEARN HOW TO USE THIS ONE BECAUSE DOUBLE INDENTING IS A FUCKING NIGHTMARE (or just import from mathcha?)
%% Example algorithm:
% \begin{center}
% 	\begin{minipage}{0.5\linewidth} % Adjust the minipage width to accomodate for the length of algorithm lines
% 		\begin{algorithm}[H]
% 			\KwIn{$(a, b)$, two floating-point numbers}  % Algorithm inputs
% 			\KwResult{$(c, d)$, such that $a+b = c + d$} % Algorithm outputs/results
% 			\medskip
% 			\If{$\vert b\vert > \vert a\vert$}{
% 				exchange $a$ and $b$ \;
% 			}
% 			$c \leftarrow a + b$ \;
% 			$z \leftarrow c - a$ \;
% 			$d \leftarrow b - z$ \;
% 			{\bf return} $(c,d)$ \;
% 			\caption{\texttt{FastTwoSum}} % Algorithm name
% 			\label{alg:fastTwoSum}   % optional label to refer to
% 		\end{algorithm}
% 	\end{minipage}
% \end{center}



%%%%%%%%%%%%%%%%%%%%%%%%%%%%%%%%%%%%%%%%%%%%%%%%%%%%%%%%%%%%%%%%%%%%%%%%%%%%%%%%%%%%
%%% Default commands:

\renewcommand{\vec}[1]{\mathbf{#1}}
	
\newcommand{\WidestEntry}{$lon_1$}%
\newcommand{\SetToWidest}[1]{\makebox[\widthof{\WidestEntry}]{#1}}%
\newcommand\tab[1][0.61cm]{\hspace*{#1}}
\newcommand{\nats}{\mathbb{N}}
\newcommand{\rats}{\mathbb{Q}}
\newcommand{\reals}{\mathbb{R}}
\newcommand{\Z}[1]{\mathbb{Z}_{#1}}
\newcommand{\BigO}[1]{\mathcal{O}(#1)}
\newcommand{\seq[1]}{(#1_n)}
\newcommand{\subseq[1]}{(#1_{n_k})}
\newcommand{\Lim}[2]{\lim \limits _{#1 \to #2}}
\newcommand{\Min}[2]{\min \{#1, #2\}}
\newcommand{\inv}{^{-1}}
\newcommand{\h}{^\text{th}}
\newcommand{\lrangle}[1]{\langle #1 \rangle}
\newcommand{\abs}[1]{\left\lvert #1 \right\rvert}

\DeclarePairedDelimiter{\ceil}{\lceil}{\rceil}
\DeclarePairedDelimiter{\floor}{\lfloor}{\rfloor}
\DeclareMathOperator{\supp}{supp}
\DeclareMathOperator{\rad}{rad}
\DeclareMathOperator*{\argmin}{arg\,min}
\DeclareMathOperator*{\argmax}{arg\,max}
\DeclareMathOperator*{\Var}{Var}
\DeclareMathOperator*{\Cov}{Cov}
\DeclareMathOperator*{\Corr}{Corr}
\DeclareMathOperator*{\Aut}{Aut}
\newcommand{\prob}[1]{\section*{Problem #1}}


%%%%%%%%%%%%%%%%%%%%%%%%%%%%%%%%%%%%%%%%%
%%% Commands I've added:

%% For Math 300 (Winter 2021):

\newcommand{\lrbrace}[1]{\{ #1 \}}
\newcommand{\powerset}{\mathcal{P}}
\newcommand{\ints}{\mathbb{Z}}

% Source/inspiration: https://tex.stackexchange.com/a/42728:
\newcommand{\numberthis}{\addtocounter{equation}{1}\tag{\theequation}\label{\theequation}}
    % Within an `align*` environment, put `\numberthis` after a line to number it. 
    % Access it with `\eqref{ [number of equation] }`
\newcommand{\numberthiswith}[1]{\addtocounter{equation}{1}\tag{\theequation}\label{#1}}
    % Within an `align*` environment, put `\numberthiswith{ [your_label] }` after a line to number it. 
    % Access it with `\eqref{ [your_label] }`



%%%%%%%%%%%%%%%%%%%%%%%%%%%%%%%%%%%%%%%%%%%%%%%%%%%%%%%%%%%%%%%%%%%%%%%%%%%%%%%%%%%%
%%% Default formatting:

\hypersetup{
    colorlinks=true,
    linkcolor=blue,
    filecolor=magenta,      
    urlcolor=blue,
}

\setlength{\parindent}{0cm}
\setlength{\parskip}{6pt}

\pagestyle{fancy}

% Modified version of problem environment below
% \newenvironment{problem}[2][Problem]
%     { \begin{mdframed}[backgroundcolor=gray!5] \textbf{#1 #2} \\}
%     {  \end{mdframed}}
% \newenvironment{solution}{\textbf{Solution}\\}


%%%%%%%%%%%%%%%%%%%%%%%%%%%%%%%%%%%%%%%%%
%%% Formatting I've added:

% Grey boxes for problem statements (note that I don't have a "solution" section):

% Problem environment, but shows "([input_char])" instead of "Problem [input_char]"
\newenvironment{problem}[2][]
    { \begin{mdframed}[backgroundcolor=gray!5] \textbf{#1 (#2)}}
    {  \end{mdframed}}
% Problem environment, but no "([input_char])" at all
\newenvironment{problem*}
    { \begin{mdframed}[backgroundcolor=gray!5] \\}
    {  \end{mdframed}}



%%%%%%%%%%%%%%%%%%%%%%%%%%%%%%%%%%%%%%%%%%%%%%%%%%%%%%%%%%%%%%%%%%%%%%%%%%%%%%%%%%%%
% Misc things I've added




%%%%%%%%%%%%%%%%%%%%%%%%%%%%%%%%%%%%%%%%%%%%%
% Fill in the appropriate information below
\lhead{Zain Kamal}
\rhead{Math 300 Winter 2021} 
% \chead{\textbf{Homework 2}} % Moved to document
\begin{document}
\chead{\textbf{Homework 3}}
%%%%%%%%%%%%%%%%%%%%%%%%%%%%%%%%%%%%%%%%%%%%%%%%%%%%%%%%%%%%%%%%%%%%%%%%%%%%%%%%%
\section{}


\begin{problem} a
Prove that for any real $x$, if $x^2 < 73$, then $0 < 1$.
\end{problem}

% This is true by trivial proof, as $0 < 1$ will always be true.

$0 < 1$ will always be true, so the conclusion is true regardless of the premises. Thus the implication is trivial.

\

% ask them if we have to prove x^2

\begin{problem} b
Prove that for any integer $x$, if $-x^2 > 0$, then $x = 5$.
\end{problem}

% This is true by vacuous proof, as $(-x^2 >0)$ will never be true for any $x$.

$(-x^2 >0)$ will never be true for any $x$, so the premise is always false. Thus the implication is vacuous.

\
\hline
%%%%%%%%%%%%%%%%%%%%%%%%%%%%%%%%%%%%%%%%%%%%%%%%%%%%%%%%%%%%%%%%%%%%%%%%%%%%%%%%%
\section{}

\begin{problem*}
Prove the following by \textbf{direct} proof.
\end{problem*}

\begin{problem} a
Prove that if $x$ is an odd integer, then $7x - 5$ is even.
\end{problem}

Let $x$ be an odd integer. Then there exists some integer $k$ such that $x=2k+1$.
% Let $x \in \ints$ such that $x=2k+1$ for some $k \in \ints$ (as per the definition of an odd integer). 
% Let $x=2k+1$ for some $k \in \ints$ (as per the definition of an odd integer). 
% Substituting $x=2k+1$ into $7x-5$ and factoring yields $2(7k+1)$. 
We can substitute and do some basic arithmetic to find:
\begin{align*}
    7x-5 &= 7(2k+1) - 5 \\
    &= 14k+2 \\
    &= 2(7k+1).
\end{align*}
We can deduce that $(7k+1) \in \ints$ because the multiplication and addition of integers can only yield an integer. Therefore $2(7k+1)$ can also be expressed as $2(k')$ for some $k' \in \ints$. This is the definition of an even integer, so we can conclude that if $x$ is an odd integer, then $7x-5$ is even.

\

\begin{problem} b
Let $a, b, c$ be integers. Prove that if $a$ and $c$ are odd, then $ab + bc$ is even.
\end{problem}

Let $a$, $b$, $c$ be integers, and let $a$ and $c$ be odd. Then there exist some integers $k_1$ and $k_2$ such that $a=2k_1+1$ and $c=2k_2+1$.
% Let $a=2k_1+1$ and $c=2k_2+1$ for some $k_1,k_2 \in \ints$ (as per the definition of odd integers).
We can substitute and do some basic arithmetic to find:
\begin{align*}
    ab+bc &= (2k_1+1)b+b(2k_2+1) \\
    &= b(2k_1+2k_2+2) \\
    &= 2(b(k_1+k_2+1)).
\end{align*}
% Substuting these values into $ab+bc$ and factoring yields $2(b(k_1+k_2+1))$. 
We can deduce that $(b(k_1+k_2+1)) \in \ints$ because the multiplication and addition of integers can only yield an integer. Therefore $2(b(k_1+k_2+1))$ can also be expressed as $2(k')$ for some $k' \in \ints$. This is the definition of an even integer, so we can  conclude that if $a$ and $c$ are odd, then $ab+bc$ is even.

\

\begin{problem} c
Prove that every odd integer is the difference of two square integers.
\end{problem}

Let $x$ be an odd integer. Then there exists some integer $k$ such that $x=2k+1$.
% Let $x \in \ints$ such that $x=2k+1$ for some $k \in \ints$ (as per the definition of odd integers). 
We can do some basic arithmetic to find:
\begin{align*}
    x &= 2k+1 \\
    &= 2k + 1 + k^2 - k^2 \\
    &= (k^2 + 2k + 1 ) - k^2 \\
    &= (k+1)^2 - k^2.
\end{align*}

We see that $k+1$ and $k$ are integers, so we can conclude that that any odd integer $x$ can be represented as the difference of two square integers.

\
\hline
%%%%%%%%%%%%%%%%%%%%%%%%%%%%%%%%%%%%%%%%%%%%%%%%%%%%%%%%%%%%%%%%%%%%%%%%%%%%%%%%%
\section{}

\begin{problem} a
Prove that for an integer $x$, $x$ is odd if and only if $x^3$ is odd. [Hint: You have to prove two directions for this, $x$ is odd $\Rightarrow$ $x^3$ is odd, and $x$ is odd $\Leftarrow$ $x^3$ is odd. It is customary to do this in two stages, where you label the direction you are proving with the relevant arrow. I would start the first stage by writing $(\Rightarrow)$, and the second stage with a $(\Leftarrow)$ to make it clear what I am doing. Also, for the second stage, the contrapositive may help you.]
\end{problem}

($\Rightarrow$) Let $x$ be an odd integer. Then there exists some integer $k$ such that $x=2k+1$. We can substitute and do some basic arithmetic to find:
\begin{align*}
    x^3 &= (2k+1)^3 \\
    &= 8 k^3 + 12 k^2 + 6 k + 1 \\
    &= 2(4k^3+6k^2+3k)+1.
\end{align*}
We can deduce that $(4k^3+6k^2+3k) \in \ints$ because the multiplication and addition of integers can only yield an integer. Therefore $2(4k^3+6k^2+3k)+1$ can also be expressed as $2(k')+1$ for some $k' \in \ints$. This is the definition of an odd integer, so we can  conclude that if $x$ is odd, then $x^3$ is odd.

\

($\Leftarrow$) We will prove that $x^3$ is odd implies that $x$ is odd by proving the contrapositive: if $x$ is even, then $x^3$ is even. Let $x$ be an even integer. This means there exists some integer $k$ (no relation to previous section) such that $x=2k$. We can substitute and do some basic arithmetic to find:
\begin{align*}
    x^3 &= (2k)^3 \\
    &= 8k^3 \\
    &= 2(4k^3).
\end{align*}
We can deduce that $4k^3 \in \ints$ because the multiplication and addition of integers can only yield an integer. Therefore $2(4k^3)$ can also be expressed as $2(k')+1$ for some $k' \in \ints$ (no relation to previous section). This is the definition of an even integer, so we can  conclude that if $x$ is even, then $x^3$ is even. Thus, by contrapositive, if $x^3$ is odd then $x$ is also odd.

\

Having proven both directions of implication, we can conclude that an integer $x$ is odd if and only if $x^3$ is odd.

\

\begin{problem} b
Consider the following definition: 

\textbf{Definition 3.19}: 
If $x$ \textbf{divides} $y$, then there is some integer $k$ such that $y = kx$.

Prove that if four does not divide $x^2$ when $x$ is an integer, then $x$ is odd. 
\end{problem}

We will proceed by proving the contrapositive: if $x$ is an even integer, then $4$ divides $x^2$. Let $x$ be an even integer. Then there exists some integer $k_1$ such that $x=2k_1$. We check if $4$ divides $x^2$ by substituting them into definition 3.19 and solving for $k$:
\begin{align*}
    (x^2) &= k(4) \\
    (2k_1)^2 &= 4k \\
    % 4k_1^2 &= 4k \\
    k_1^2 &= k.
\end{align*}
We observe that $k$ is an integer, so we can conclude that if x is an even integer, then 4 divides $x^2$ by definition 3.19. Thus by contrapositive, if 4 does not divide $x^2$, then $x$ is an odd integer.

\
\hline
%%%%%%%%%%%%%%%%%%%%%%%%%%%%%%%%%%%%%%%%%%%%%%%%%%%%%%%%%%%%%%%%%%%%%%%%%%%%%%%%%
\section{}

\begin{problem} a
Prove that there is no largest integer.
\end{problem}

Assume for the sake of contradiction that there does exist some largest integer $x$ such that it is greater than all integers. Let $y=x+1$ be an integer. We observe that $y>x$, which contradicts our assumption that $x$ is greater than all integers. Therefore there exists no largest integer.

\

\begin{problem} b
Prove that there is no smallest positive rational number.
\end{problem}

Assume for the sake of contradiction that there does exist some smallest positive rational number $x$. Let $y=x/2$ be a rational number. We observe that $y<x$ and $y>0$, which contradicts our assumption that $x$ is less than all positive rational numbers. Therefore there exists no smallest positive rational number.

\

\begin{problem} c
Prove or disprove: The product of two irrational numbers is irrational.
\end{problem}

Let $x$ be a rational number such that $\sqrt{x}$ is irrational. We observe that $\sqrt{x} \cdot \sqrt{x} = x$, proving tht the product of two irrational numbers could be irrational.

\

\begin{problem} d
Prove or disprove: The sum of a rational and irrational number is irrational.
\end{problem}

Let $p/q$ be a rational number and let $x$ be an irrational number. For the sake of contradiction, let their sum be a rational number, $p'/q'$. We can rearrange this statement to find:
\begin{align*}
    \frac{p}{q} + x &= \frac{p'}{q'} \\
    x &= \frac{p'}{q'} - \frac{p}{q} \\
    x &= \frac{p'q - pq'}{qq'}.
\end{align*}

We observe that $p'q$, $pq'$, and $qq'$ all yield integers, so $x$ can be expressed as a ratio of nonzero integers. But this contradicts our premise that $x$ is an irrational number. Therefore the sum of a rational and irrational number must be irrational.

\
\hline
%%%%%%%%%%%%%%%%%%%%%%%%%%%%%%%%%%%%%%%%%%%%%%%%%%%%%%%%%%%%%%%%%%%%%%%%%%%%%%%%%
\section{}

\begin{problem} a
\textbf{(Triangle Inequality).} Prove that for any $x,y\in\mathbb{R},$ $|x+y|\le |x|+|y|$. [Hint: Proof by cases.]
\end{problem}

Let $x$ and $y$ be real numbers. Then we have the following cases:
\begin{itemize}
    \item \textbf{Case 1:} Let $|x+y|=x+y$. From this, we can simplify:
    \begin{align*}
        |x+y| & \leq |x|+|y| \\
        x+y & \leq |x|+|y|,
    \end{align*}
    which is necessarily true because $x\leq |x|$ and $y\leq |y|$. Therefore the Triangle Inequality holds for the case $|x+y|=x+y$.
    
    \item \textbf{Case 2:} Let $|x+y|=-(x+y)$. From this, we can simplify:
    \begin{align*}
        |x+y| & \leq |x|+|y| \\
        -x-y & \leq |x|+|y|,
    \end{align*}
    which is necessarily true because $-x \leq |x|$ and $-y \leq |y|$. Therefore the Triangle Inequality holds for the case $|x+y|=-(x+y)$.
    % \item \textbf{Case 1:} Let $x>0$ and $y>0$. Because both of these values are positive, we can simplify:
    % \begin{align*}
    %     |x+y| & \leq |x|+|y| \\
    %     x+y & \leq x+y.
    % \end{align*}
    % This is a tautology, so the Triangle Inequality holds for the case $x>0$ and $y>0$.
    
    % \item \textbf{Case 2:} Let $x>0$ and $y<0$. Because $x$ is positive, we can simplify:
    % \begin{align*}
    %     |x+y| & \leq |x|+|y| \\
    %     |x+y| & \leq x+|y|.
    % \end{align*}
    % We then solve for both cases of the inequality. In the first case,
    % \begin{align*}
    %     x+y & \leq x+|y| \\
    %     y & \leq |y|,
    % \end{align*}
    % which is necessarily true. In the second case,
    % \begin{align*}
    %     x+y & \geq -x-|y| \\
    %     2x+y-|y| & \geq 0\\ 
    %     x \geq 0,
    % \end{align*}
    % which is necessarily true given our assumption $x>0$. Therefore the Triangle Inequality holds for the case $x>0$ and $y<0$.
    
    % \item \textbf{Case 3:} Let $x<0$ and $y>0$. The Triangle Inequality holds true for this case because of its symmetry with case 2, with $x$ and $y$ switched.
    
    % \item \textbf{Case 4:} Let $x<0$ and $y<0$. Because both of these values are negative, we can simplify:
    % \begin{align*}
    %     |x+y| & \leq |x|+|y| \\
    %     x+y & \leq x+y.
    % \end{align*}
    
\end{itemize}

The Triangle Inequality holds for both cases, so we can conclude that it is true.

\

\begin{problem} b
\textbf{(Reverse Triangle Inequality).} Prove that for any $x,y\in\mathbb{R},$ $||x|-|y||\le |x-y|.$ [Hint: $x=x-y+y$.]
\end{problem}

Let $x$ and $y$ be real numbers. Then we have the following cases:
\begin{itemize}
    \item \textbf{Case 1:} Let $||x|-|y||=|x|-|y|$. We can prove this by expanding $|x|$:
    \begin{align*}
        |x| &= |x+y-y| \\
        &= |(x-y) + (y) |,
    \end{align*}
    at which point we can apply the Triangle Inequality to find:
    \begin{align*}
        |x| &\leq |x-y| + |y| \\
        |x| - |y| &\leq |x-y|.
    \end{align*}
    Thus the Reverse Triangle Inequality holds for $||x|-|y||=|x|-|y|$.
    
    \
    
    \item \textbf{Case 2:} Let $||x|-|y||=-(|x|-|y|)=|y|-|x|$. We repeat the same process from case 1 but replacing $x$ and $y$ to find $|y| - |x| \leq |x-y|$. Thus the Reverse Triangle Inequality holds for $||x|-|y||=-(|x|-|y|)$.
    
\end{itemize} 

The Reverse Triangle Inequality holds for both cases ($\big| |x|-|y| \big|=\pm (|x|-|y|)$), so we can conclude that it is true.



%%%%%%%%%%%%%%%%%%%%%%%%%%%%%%%%%%%%%%%%%%%%%%%%%%%%%%%%%%%%%%%%%%%%%%%%%%%%%%%%%
\end{document}
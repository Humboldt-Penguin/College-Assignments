%%%%%%%%%%%%%%%%%%%%%%%%%%%%%%%%%%%%%%%%%
% Lachaise Assignment
% LaTeX Template
% Version 1.0 (26/6/2018)
%
% This template originates from:
% http://www.LaTeXTemplates.com
%
% Authors:
% Marion Lachaise & François Févotte
% Vel (vel@LaTeXTemplates.com)
%
% License:
% CC BY-NC-SA 3.0 (http://creativecommons.org/licenses/by-nc-sa/3.0/)
% 
%%%%%%%%%%%%%%%%%%%%%%%%%%%%%%%%%%%%%%%%%

%----------------------------------------------------------------------------------------
%	PACKAGES AND OTHER DOCUMENT CONFIGURATIONS
%----------------------------------------------------------------------------------------

\documentclass{article}

\input{Structures/structure_1} % Include the file specifying the document structure and custom commands

%----------------------------------------------------------------------------------------
%	ASSIGNMENT INFORMATION
%----------------------------------------------------------------------------------------

\title{Math 300 -- Homework 1} % Title of the assignment

\author{Zain Kamal\\ \texttt{z.kamal2021@gmail.com}} % Author name and email address

\date{\today} % University, school and/or department name(s) and a date

%----------------------------------------------------------------------------------------

\begin{document}

\maketitle % Print the title

\section{Translation} % Numbered section

\begin{enumerate}[label=\alph*)]
    \item 
    Define $A$ as the set of everyone.

    Define $L_D(x) := x$ loves dogs.
    
    Then $\forall a \in A, L_D(a)$.
    \item 
    Define $D$ as the set of all dogs.
    
    Define $L(x,y) := x$ loves $y$.
    
    Then $\forall a \in A, (\exists d \in D : L(a, d)) \lor a \in D$.
    
    % \textit{Alternative (I think this is incorrect because I define $d$ even though $a \in D$ does not make use of $d$, but I'm not sure): }
    
    % Then $\forall a \in A, \exists d \in D : L(a, d) \lor a \in D$.
    
    % Then $\forall a \in A, L(a) \lor a \in D$
    
    % Not good, d is unnecessary: Then $\forall a \in A, \exists d\in D : L(a) \lor (a=d)$.
    
    % Alt: Then $\forall a \in A, L(a) \lor (\exists d\in D : a=d)$.
    
    \item
    % $\forall a \in \mathbb{Z}, \exists b := a + 1 > a$.
    
    $\forall a \in \mathbb{Z}, \exists b \in \mathbb{Z} : b > a$.
    
    % Let $l$ be the largest integer.
    
    % Then $\nexists ln$.
    
    \item
    Define $R(x) := x$ gets rest.
    
    Define $E(x) := x$ gets exercise.
    
    Define $G(x) := x$ has a good diet.
    
    Define $H(x) := x$ is healthy.
    
    Then $\forall a \in A, H(a) \Rightarrow R(a) \land E(a) \land G(a)$.
    
    % WRONG: Then $\forall a \in A, R(a) \land E(a) \land G(a) \Rightarrow H(a)$.
    
    \item
    $\forall a \in \mathbb{Z}, \exists b \in \mathbb{Z} : b > a$.
    
    \item
    Define $P(x)$ as true if $x$ is prime and false if not. 
    
    Define $D(a,b) := a$ divides $b$. 
    
    Let $x \in \mathbb{Z}$.
    
    $P(x) := x > 1 \land (\forall y \in \mathbb{Z}, D(y, x) \Leftrightarrow y=1 \lor y=x)$.
    
    % $P(x) := x > 1 \land (\forall y \in \mathbb{Z}, (y>1 \land y \neq x) \Rightarrow \lnot D(y, x))$.
    
    % Define $P$ as the set of prime numbers.
    
    % Let $x,y \in \mathbb{R}$.
    
    % Define $D(a,b) := a$ divides $b$.
    
    % % Then $x\in P \Leftrightarrow x>1 \land (\nexists y \in \mathbb{R} : y \neq 1 \land D(y,x))$.
    
    % Then $x\in P \Leftrightarrow x>1 \land (\forall y \neq 1, \lnot  D(y,x))$.
    
    % % $\forall y \in \mathbb{R}, y \neq 1 \Rightarrow \varepsilon $
    
    
\end{enumerate}
    
%----------------------------------------------------------------------------------------

\section{Negation}

\begin{enumerate}[label=\alph*)]
    \item 
    $\exists a \in A : \lnot L_D(a)$.
    
    \item
    $\exists a \in A : (\forall d \in D, \lnot L(a,d)) \land a \notin D$.
    % NEXISTS IS NOT A THING, ONLY LNOT EXISTS
    
    % $\exists a \in A : (\nexists d \in D : L(a,d)) \land a \notin D$.
    %% (\forall d \in D, \lnot L(a,d)) IS PREFERRABLE TO (\nexists d \in D : L(a,d))
    
    % $\exists a \in A : (\lnot \exists d \in D : L(a,d)) \land \lnot(a \in D)$.
    
    \item
    $\exists a \in \mathbb{Z} : \forall b \in \mathbb{Z}, a \geq b$.
    
    \item
    $\exists a \in A : H(a) \land (\lnot R(a) \lor \lnot E(a) \lor \lnot G(a))$.
    % NOTE THE NEGATION HERE
    
    \item
    $\exists a \in \mathbb{Z} : \forall b \in \mathbb{Z}, a \geq b$.
    
    \item
    $\lnot P(x) \Leftrightarrow x < 1 \lor (\exists y \in \mathbb{Z} : (y>1 \land y \neq x) \Rightarrow D(y,x))$. 
    % If the statement were $x>1 \land (\forall y \neq 1, \lnot  D(y,x))$,
    
    % then the negation would be $x \leq 1 \lor (\exists y \neq 1 : D(y,x))$. \\
    
    % However, if the statement were $x\in P \Leftrightarrow x>1 \land (\forall y \neq 1, \lnot  D(y,x))$, generalized to the form $X \Leftrightarrow Y$,
    
    % then the negation would take the generalized form $(X \land \lnot Y) \lor (\lnot X \land Y)$. \\
    
    % \textbf{ASK THEM ABOUT THIS TOMORROW, I'VE ASKED TOO MANY QUESTIONS TODAY LOL.}

\end{enumerate}

%----------------------------------------------------------------------------------------

\section{Exercises}

\begin{enumerate}[label=\alph*)]
    \item 
    (LEM to Consistency via DeMorgan's Law:)
    
    We start with LEM: $(p \lor \lnot p) \Leftrightarrow T$.
    
    Negate both sides: $\lnot(p \lor \lnot p) \Leftrightarrow \lnot T$.
    
    Use DeMorgan's Law on left side: $(\lnot p \land \lnot \lnot p) \Leftrightarrow \lnot T$.
    
    Simplify left side ("double negative") and right side ("truth"): $(\lnot p \land p) \Leftrightarrow F$.
    
    Rearrange left side ("commutivity of and"): $(p \land \lnot p) \Leftrightarrow F$. We now have "consistency"!
    
    \item
    (Absorption exercise:)
    
    We start with $(p \land (p \lor q))$.
    
    Using distribution, this becomes: $(p \land p) \lor (p \land q)$.
    
    Using idempotence, this becomes: $(p \lor (p \land q))$. 
    
    Therefore $(p \land (p \lor q)) \equiv (p \lor (p \land q))$.
    
    \item
    (Explain contrapositive:)
    
    $(p \Rightarrow q)$ means that if $p$ is true, then $q$ must necessarily be true. Therefore, if we observe $q$ is false, then we know that $p$ couldn't have been true, i.e. $p$ is false. 
    
    \item
    (Modus tollens to modus ponens:)
    
    We start with modus tollens: $(\lnot q \land (p \Rightarrow q)) \Rightarrow \lnot p$.
    
    Define $a := \lnot p$ and $b := \lnot q$ and substitute to get: $(b \land (\lnot a \Rightarrow \lnot b)) \Rightarrow a$.
    
    Simplify the contrapositive on the left side: $(b \land (b \Rightarrow a)) \Rightarrow a$. This is Modus ponens.
    
\end{enumerate}

%----------------------------------------------------------------------------------------

\section{Xor}

$P \ovee Q \Leftrightarrow (P \lor Q) \land \lnot(P \land Q)$ 

$\quad \ \ \Leftrightarrow (P \lor Q) \land  (\lnot P \lor \lnot Q)$.\\


$P \ovee Q$ is true if and only if "either $P$ or $Q$ is true, but not both". 

The first part of that statement, "either $P$ or $Q$ is true," can be written as $P \lor Q$. 

The second part, "but not both [are true]," can be written as $\lnot (P \land Q)$, or $(\lnot P \lor \lnot Q)$ by DeMorgan's Law.

$P \ovee Q$ is the conjunction of the first and second statement, so we combine them with $\land$.

%----------------------------------------------------------------------------------------

\section{Exercise}

\begin{enumerate}[label=\alph*)]
    \item 
    "For all numbers $x$ in the set of real numbers, there exists a number $y$ in the set of real numbers such that the sum of $x$ and $y$ is $0$.
    
    This is true. If we let $y = -x$ and substitute into $x+y=0$, we get $x-x=0 \Leftrightarrow 0=0$.
    
    \item
    "There exists a number $x$ in the set of real numbers such that, for all numbers $y$ in the set of real numbers, the sum of $x$ and $y$ is $0$."
    
    This is false. For example, if we choose $y=0$ and substitute into $x+y=0$, we get $x=0$. However, if we instead choose $y=1$ for the substitution, we get $x=-1$. $x$ can't simultaneously be equal to $0$ and $-1$, so we have a contradiction. Therefore the initial proposition is false.

\end{enumerate}

%----------------------------------------------------------------------------------------

\section{Uniqueness}

The statement is equivalent to: $\exists x \in X : P(x) \land (\forall y \in X, P(y) \Leftrightarrow y=x)$.

This is equivalent because $\exists x \in X : P(x)$ means there's at least one element $x$ that satisfies $P$, but there could be other such elements. To rule out the existence of these other elements, we must clarify that out of all the elements in $X$, the only one that satisfies $P$ is $x$.

However, this answer is technically slightly redundant. We can get rid of $P(x)$ and get: $\exists x \in X : (\forall y \in X, P(y) \Leftrightarrow y=x)$.

%----------------------------------------------------------------------------------------
% SECTION TEMPLATE
% ----------------------------------------------------------------------------------------

% \section{Negation}

% \begin{enumerate}[label=\alph*)]
%     \item 

% \end{enumerate}

%----------------------------------------------------------------------------------------

\end{document}
